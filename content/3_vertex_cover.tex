%!TEX root = ../exact.tex

\section{Reduction to Vertex Cover}\label{sec:vertex_cover}

After applying a set of reduction rules exhaustively we solve the remaining instance by repeatedly solving vertex cover instances.
Initially, we choose the reduction rules Improved CORE and PIE.
If this doesn't solve the instance within a few seconds, we proceed by using all reduction rules listed above.

% \paragraph*{Solver Description}
First note that if the remaining graph contains only bidirectional edges, we can easily reduce the DFVS instance to a vertex cover instance by turning bidirectional edges into undirected edges.
Initially, we find an optimal vertex cover $S$ in $G[\text{PIE}]$. If $S$ is a DFVS, $S$ must be optimal.
Otherwise, we find a set of vertex-disjoint cycles $C$ in $G-S - PIE$ using a DFS.
All cycles in $C$ are not covered by $S$, so we add a gadget to each cycle to ensure, that in the modified graph, there is an optimal vertex cover, which includes a $v\in S$.
Finally, we iterate on the modified graph until the vertex cover is also a DFVS. Note, that this can happen multiple times since our choice of $C$ does not guarantee that all cycles in $G$ are covered.

% \paragraph*{Clique Gadget}
Let $G = (V,E)$ be an undirected graph and let $S \subseteq V$. Our goal is to find the minimum vertex cover in $G$ that also contains a vertex in $S$. To achieve this, we add a clique of size $|S|$ to $G$ and connect it one-to-one with $S$. We call the modified graph $G'$.
Consider any optimal vertex cover $C$ in $G'$. Then, $C$ contains at least $|S|-1$ vertices in the new clique.
Also, $C$ must cover all edges between $V$ and the clique, so it must contain at least one vertex in $S$ or all vertices in the clique.
If $C$ contains all vertices in the clique, we exchange one of these vertices for a vertex in $S$ and obtain an optimal vertex cover $C'$ in $G'$ with $C' \cap S \neq \emptyset$.
Thus, $C'\cap V$ is an optimal vertex cover of $G$ that also contains a vertex of $S$.

% \paragraph*{Speedup With Heuristic Solvers}
Note that the exactness of our solver only relies on the optimality of the final vertex cover. Therefore, when the exact solver takes too long, we switch to a heuristic solver and verify our solution with an exact solver once we have covered all cycles in $G$. If the heuristic vertex cover was not optimal, we iterate further. Surprisingly, this did not occur on any public instance.

% \paragraph*{Vertex Cover Solvers}
To solve a vertex cover instance, we initially reduce it using the kernelization procedure, implemented by the winning solver of the 2019 PACE challenge \cite{Hespe2020WeGotYouCoveredTW}.
When solving this instance heuristically, we use a local-search solver~\cite{Cai2013NuMVCAE} on this kernel.
To solve the kernel exactly, we use a branch-and-reduce solver~\cite{pg21satreduce}, which we augment by implementing better upper bounds using the aforementioned local-search solver.

