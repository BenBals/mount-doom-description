%!TEX root = ../heuristic.tex

\section{Solver Description}\label{sec:heuristic-solver}

After exhaustively applying the reductions we described in~\Cref{sec:reductions}, we produce two solutions, one by a greedy procedure and another by a reduction to vertex cover.
The better solution of both approaches is minimized further by applying 2-1 swaps until the timelimit is hit.

\subsection{Greedy Routine}\label{ssec:greedy-routine}

% The greedy routine is described more formally in~\Cref{alg:greedy-routine}.
For the initial solution, we start with an empty set and greedily add to it the vertex of highest degree until we obtain a feasible DFVS. Checking for feasibility here means searching for a cycle in the graph. To speedup this procedure, we only search for a new cycle, once the old one is covered.
Additionally, after a fixed number of nodes are taken into the solution, we reapply all reduction rules.
After the initial solution is generated, we remove nodes that do not reintroduce a cycle when added back to the graph. This ensures that we create an inclusion-minimal solution.

\subsection{Reduction to Vertex Cover}\label{ssec:vertex_cover}

% \paragraph*{Solver Description}
First note that if a graph contains only bidirectional edges, we can easily reduce the DFVS instance to a vertex cover instance by turning bidirectional edges into undirected edges.
Initially, we find a vertex cover $S$ in $G[\text{PIE}]$ and check, whether $S$ is a DFVS for $G$.
If not, we find a set of vertex-disjoint cycles $C$ in $G-S - PIE$ using a DFS.
All cycles in $C$ are not covered by $S$, so we add a gadget to each cycle to ensure, that in the modified graph, there is an optimal vertex cover, which includes a $v\in S$.
Finally, we iterate on the modified graph until the vertex cover is also a DFVS. Note, that this can happen multiple times since our choice of $C$ does not guarantee that all cycles in $G$ are covered.
When we hit an internal timelimit before finding a feasible DFVS, we apply our greedy approach described in \Cref{ssec:greedy-routine} for the remaining graph.

% \paragraph*{Clique Gadget}
Let $G = (V,E)$ be an undirected graph and $S \subseteq V$ be a cycle. Our goal is to find the minimum vertex cover in $G$ that also contains a vertex in $S$. To achieve this, we add a clique of size $|S|$ to $G$ and connect it one-to-one with $S$. We call the modified graph $G'$.
Consider any optimal vertex cover $C$ in $G'$. Then, $C$ contains at least $|S|-1$ vertices in the new clique.
Also, $C$ must cover all edges between $V$ and the clique, so it must contain at least one vertex in $S$ or all vertices in the clique.
If $C$ contains all vertices in the clique, we exchange one of these vertices for a vertex in $S$ and obtain an optimal vertex cover $C'$ in $G'$ with $C' \cap S \neq \emptyset$.
Thus, $C'\cap V$ is a optimal vertex cover of $G$ that also contains a vertex of $S$.

To solve the vertex cover instance, we first use a kernelization procedure implemented by the winning solver of the 2019 PACE challenge \cite{Hespe2020WeGotYouCoveredTW}.
Then, we use a local-search solver~\cite{Cai2013NuMVCAE} on this kernel.

%\begin{algorithm}
%    \DontPrintSemicolon
%    \SetAlgoLined
%    \LinesNumbered
%    \KwData{Graph \(G=(V, E)\), reduction interval \(k\)}
%    \KwResult{DFVS \(S\subseteq V\)}
%    reduce \(G\) exhaustively\;
%    \(S\leftarrow\{\}\), \(G'\leftarrow G\)\;
%    \tcp{find an initial solution}
%    \While{\(G'\) has a cycle \(C\subseteq V'\)}{
%        \(v\leftarrow \text{argmax}_{u\in C}{|N_{G'}^+(u)| + |N_{G'}^-(u)|}\)\;
%        \(S\leftarrow S\cup\{v\}\)\;
%        \(G'\leftarrow G'-\{v\}\)\;
%        \If{\(|S|\) is a multiple of \(k\)}{
%            reduce \(G'\) exhaustively\;
%        }
%    }
%    \tcp{remove non-essential nodes}
%    \For{\(v\in S\)}{
%        \If{\(G-(S\backslash \{v\})\) has no cycle}{
%            \(S\leftarrow S\backslash \{v\}\)\;
%        }
%    }
%    improve solution using 2-1 swaps (optional)\;
%
%    \caption{Abstract description of greedy routine.}\label{alg:greedy-routine}
%\end{algorithm}

\subsection{2-1 swaps}
We apply a local-search approach to improve our current solution, named 2-1 swaps. As the name suggests, its goal is to find 2 vertices from a given DFVS-solution and replace them with one vertex to form a smaller solution. The idea uses the notion of skew separator from Chen et al.~\cite{Chen2008AFA}. 

Consider a feasible and minimal solution set $S$ to the DFVS problem, i.e.~for all $v \in S$ the set $S - v$ is infeasible. 
Our goal is to find $v,w\in S$ and $u\notin S$ such that $\left(S-\{v,w\}\right)\cup \{u\}$ is a feasible solution.
Whenever we find such a triple $(v,w, u)$, we swap $\{v, w\}$ with $u$ in $S$ and repeat the procedure.

We now describe how to find such a triple.
First, for each vertex $v \in S$, we find all vertices $u \in G-S$, called candidates, such that $\left(S-\{v\}\right)\cup \{u\}$ remains feasible. 
Let the candidates of a given solution vertex $v$ be $C_v$. 

In order to find $C_v$, we split the vertex $v$ into two vertices $v_{\text{out}}, v_{\text{in}}$ with out- or ingoing edges of $v$ only. Let $G^v:= (G-S)\cup \{v_{out}, v_{in}\}$.  Note that there is no edge between $v_{\text{out}}$ and $v_{\text{in}}$ in either direction in $G^v$. The candidates are now the set of all vertices $x$ such that $x$ hits all $v_{out}\leadsto v_{in}$ paths in $G^v$.

Note that in $G-S$, the vertices in $C_v$  have a topological ordering.
Also, notice that each cycle in $(G-S)\cup \{v\}$ has to be hit by each vertex in $C_v$.
%Also, there exist at least one cycle in $(G-S) \cup \{v\}$ by minimality of $S$.
This implies that $C_v$ has a unique topological ordering in $G-S$.
Let $\tau_v$ be this ordering.
% Note that in $G-S$, these candidates have a topological ordering and since all candidates for a given solution vertex cover the same set of cycles, this ordering is unique.

We say that the ordered triple $(v,w,u)$ is a swap triple if there is no $v_{out}\leadsto v_{in}$ path in $G^v$, there is no $w_{out}\leadsto w_{in}$ path in $G^w$, and there is no $v\leadsto w$ path in $(G-\left(S\cup\{u\}\right))\cup \{v,w\}$.
Observe that a valid 2-1 swap is possible if and only if there exists such a swap triple.
For each vertex $v$, we now find a $w$ and $u$ that give a swap triple $(v,w,u)$ if such a $w$ and $u$ exist, as follows.
Observe that $(v,w,u)$ is a swap triple if and only if $u\in C_v\cap C_w$ and $u$ hits all $v\leadsto w$ paths in $(G-S)\cup \{v,w\}$.
Let $W$ be the set of all vertices not reachable from $v$ in $G-C_v$.
For each $w\in W$, let $c_w$ be the first vertex in the ordering $\tau_v$ such that there is a $c_w\leadsto w$ path in $(G-C_v)\cup \{c_w\}$.
If there is a vertex $u$ in $C_v\cap C_w$ such that $u\preceq c_w$ in $\tau_v$ then we return $(v,w,u)$ as a swap triple.

First, we prove that such a $(v,w,u)$ is indeed a swap triple.
Suppose this is not the case.
Since $u\in C_u\cap C_w$, this means that 
there is a $v\leadsto w$ path $P$ in $(G-S)\cup\{v,w\}$ not hit by $u$.
The path $P$ contains at least one vertex from $C_v$ as $w$ was not reachable from $v$ in $G-C_v$.
Let $c$ be the last vertex in $P$ that is from $C_v$.
Then there is a $c\leadsto w$ path in $(G-C_v)\cup\{c\}$,
and hence $c_w\preceq c$ by the choice of $c_w$.
This implies that $u\preceq c$ by choice of $u$.
Let $P_1$ be the sub-path $v\leadsto c$ of $P$.
Since $u$ does not hit $P$, the path $P_1$ does not contain $u$.
We know there is at least one cycle in $(G-S)\cup \{v\}$ containing $C_v\cup \{v\}$, and hence there should be a $c\leadsto v$ path in $(G-S)\cup \{v\}$, say $P_2$.
The concatenation $P_1\cup P_2$ gives a cycle in $G-S\cup\{v\}$ and hence needs to be hit by each vertex in $C_v$, in particular $u$.
This means that $u$ should hit $P_2$, but this contradicts that $u\preceq c$ in $\tau_v$.


Next, we prove that if there is a swap triple then our procedure will find one.
Suppose we do not find a swap triple and there is a swap triple $(v,w,u)$.
Let us consider the processing of $u$ in our procedure.
We have $w\in W$ as $u$ hits all $v\leadsto w$ paths in $(G-S)\cup \{v,w\}$.
If $u\preceq c_w$ then since $u\in C_v\cap C_w$, we would have found the swap triple  $(v,w,u)$.
Thus $c_w\prec u$.
Then $u$ cannot be in any $v\leadsto c_w$ path in $(G-S)\cup \{v,w\}$.
But then $u$ has to hit all $c_w\leadsto w$ paths in $(G-S)\cup \{v,w\}$ as $u$ hits all $v\leadsto w$ paths in $(G-S)\cup \{v,w\}$.
This is a contradiction to the choice of $c_w$.

%\subsection{2-1 swaps}
%We apply another greedy approach to improve our current solution, named 2-1 swaps. As the name suggests, its goal is to find 2 vertices from a given DFVS-solution and replace them with a vertex to form a smaller solution. The idea uses the notion of skew separator from Chen et al. \cite{Chen2008AFA}. 
%
%Consider a feasible and minimal solution set $S$ to the DFVS problem, i.e. for all $v \in S$ the set $S - v$ is infeasible. 
%Our goal is to find a triple of vertices $(v,w,u)$ where $v,w\in S$ and $u\notin S$ such that $\left(S-\{v,w\}\right)\cup \{u\}$ is a feasible solution.
%First, for each vertex $v \in S$, we find all vertices $u \in G-S$, called candidates, such that $\left(S-\{v\}\right)\cup \{u\}$ remains feasible. 
%Let the candidates of a given solution vertex $v$ be $C_v$. 
%
%In order to find $C_v$, we split the vertex $v$ into two vertices $v_{\text{out}}, v_{\text{in}}$ with out- or ingoing edges of $v$ only. Note that there is no edge between $v_{\text{out}}$ and $v_{\text{in}}$ in either direction. The candidates are now separator vertices for all $v_{out} \leadsto v_{in}$ paths in  $(G-S)\cup \{v_{out}, v_{in}\}$. 
%
%Note that in $G-S$, the vertices in $C_v$  have a topological ordering and since there is at least one cycle that is containing  $C_v$, the ordering is unique.
%% Note that in $G-S$, these candidates have a topological ordering and since all candidates for a given solution vertex cover the same set of cycles, this ordering is unique.
%In $G-C_v$, we then mark all vertices $u$ not reachable from $v$, with the first candidate (in increasing topological order) $c$ of $C_v$ so that $c \leadsto u$ in $G-C_v \cup \{c\}$, meaning that all candidates $c' \preceq c$ in $C_v$ cover all paths $v \leadsto u$. 
%%For another solution vertex $w \in S, w \neq v$ we can then check if any node $u \in C_w$ is also present in $C_v$ and if $u$ is before the first candidate of $C_v$ that reaches $w$, as computed above. 
%
%Let $w \in S, w \neq v$ be another solution vertex and $c_w \in C_v$ be the node that marked $w$. We can then check if there exists a node $y \in C_w \cap C_v$ for which $y \preceq c_w$ in the topological ordering of $C_v$ holds, meaning that all $v \leadsto w$ paths have the subpath $y \leadsto c_w$. 
%Therefore, $y$ prevents all paths from $v_{out} \leadsto v_{in}, w_{out}\leadsto w_{in}$, and $v_{out} \leadsto w_{in}$ in $G-S \cup \{v, w\}$, covering all cycles with $v$, $w$, or $v$ and $w$. Thereby, $S - \{v, w\} \cup \{y\}$ is still a feasible solution.
%
%Note that this procedure is conducted for both sides, i.e. the procedure will also take place for swapped roles of $v$ and $w$.